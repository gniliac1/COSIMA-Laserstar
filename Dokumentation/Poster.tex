\documentclass{sciposter}
\usepackage{lipsum}
\usepackage{epsfig}
\usepackage{amsmath}
\usepackage{amssymb}
\usepackage{multicol}
\usepackage{graphicx,url}
\usepackage{textpos}   
\usepackage[utf8]{inputenc}
\usepackage{xcolor}
\newtheorem{Def}{Definition}



\title{
\leavevmode{\includegraphics[width=0.5\textwidth]{Logos/gestikulaser.png}}\\
Gestikulaser}
%Título do projeto

\author{Christoph Behr, Cailing Fu, Nicole Grubert, Daniel Wolff}
%nome dos autores by the current \maketitle

%\renewcommand\printleftlogo
%  {\begin{center}
%     \resizebox{\textwidth}{!}%
%       {\includegraphics{TOSLogo.png}\includegraphics{COSIMALogo.png}}
%   \end{center}
%  }
  
%\rightlogo[2]{TOSLogo.png}
%\leftlogo[2]{COSIMALogo}

%Section title color:
\definecolor{SectionCol}{rgb}{0.0, 0.329411, 0.62353} % RWTH Blau
% Section block color:
\definecolor{BoxCol}{rgb}{0.949,0.949,0.949} % Grau

\begin{document}

% TOS Logo oben links
\begin{textblock*}{60px}(0cm,0cm)
\includegraphics[height=6cm,width=20cm]{Logos/TOS.png}
\end{textblock*}

% COSIMA18 Logo oben rechts
\begin{textblock*}{60px}(55cm,0mm)
\includegraphics[height=6cm,width=20cm]{Logos/Cosima18.png}
\end{textblock*}

\maketitle

%%% Begin of Multicols-Enviroment
\begin{multicols}{3}
\setlength{\parindent}{2em}

\section{Unsere Vision}
Gestenerkennung ist immer wieder ein Thema, welches viel Aufmerksamkeit erregt. Und obwohl der Mensch so einfach typische Gesten erkennen kann, bleibt es für den Computer eine große Herausforderung, zuverlässig die Gesten eines Menschen zuzuordnen. \\
Mit Gestikulaser wollen wir ein neues System entwickeln, um die Handgesten eines Menschen zu erkennen und diese individuell auf den Nutzer ab zu stimmen. Dabei soll es nicht, wie die meisten heute typischen Systeme, mit einer Kamera arbeiten, sondern durch Lichtreflexionen einer Hand die Geste zuordnen. Denn damit ist die Gestenerkennung nicht nur Tageslicht unabhängig, sondern kann auch in vollkommener Dunkelheit betrieben werden. \\

\section{Der Gestikulaser}
Der Gestikulaser besteht aus zwei Teilen, einer Photoplatte sowie einer Machine Learning Software. \\
Die Photoplatte ist ausgestattet mit mehreren infrarot LED Quellen, sowie einer Vielzahl von Photodioden, welche ausschließlich infrarotes Licht detektiert. Durch die Lichtreflexion der Hand oberhalb der Photoplatte können verschiedene Intensitäten an den Photodioden gemessen werden. Durch diese Intensitäten soll dann mit Hilfe der Machine Learning Software eine eindeutige Geste erkannt werden. Diese Geste kann dazu genutzt werden verschiedene Dinge wie ein Auto oder eine Smart Home Einrichtung zu steuern.


\section{Die Beta-Version}
Foto von alter Photoplatte wo alles noch drauf geklebt ist etc.

%%% Introduction
\section{Photoplatte}
Beschreibung zur Photoplatte \\

% Einfügen eines Bildes
%\begin{figure}[h]
%\begin{center}
%\includegraphics[width=10cm]{TOSLogo.png}
%\end{center}
%\end{figure}

\section{Oktokommander}
Hier kommt ein Text mit Bildern zum Oktokommander und wie dieser funktioniert.

\section{Detektormodul}
Hier werden die Detektormodule genau beschrieben.

\section{Ausblick}
Der Handschuh und das ganze modulare Zeug beschreiben.

\section{Sponsoren}

\end{multicols}

\conference{\raisebox{0.5cm}[0cm]{\includegraphics[height=3cm]{Logos/VDE.jpg} } \hfill
\raisebox{1.25cm}[0cm]{\includegraphics[height=1.5cm]{Logos/Faulhaber.png}} \hfill
\raisebox{0cm}[0cm]{\includegraphics[height=4cm]{Logos/micronit.jpg}} \hfill
\raisebox{0cm}[0cm]{\includegraphics[height=4cm]{Logos/electronica.png}} \hfill 
\raisebox{0cm}[0cm]{\includegraphics[height=4cm]{Logos/BMBF.jpg}}}

\end{document}