\chapter{Marketing und Kosten}
\label{ch:Kosten}

Neben der Entwicklung unseres Produktes wurde im Laufe der Entwicklungszeit zusätzliche Öffentlichkeitsarbeiten geleistet. So wurde die Website \url{https://www.gestikulaser.de/} ins Leben gerufen, sowie der Instagram Account \href{https://www.instagram.com/laserharptos/}{\texttt{laserharptos}} mit über 100 Abonnenten betreut. In Zusammenhang mit diesen Marketing Maßnahmen wurden auch Sponsoren angeworben, welche Interesse an der Unterstützung unseres Projektes haben. \\
Folgende Sponsoren haben uns bei diesem Projekt unterstützt: \\

\begin{tabularx}{\textwidth}{r c}
	Aconity3D GmbH & \noindent\parbox[c]{\hsize}{\includegraphics[scale=0.07]{../Logos/AC3D_Logo_Print-for-white.png}} \\
	& \\
	Fraunhofer ILT & \noindent\parbox[c]{\hsize}{\includegraphics[scale=0.1]{../Logos/Fraunhofer_ILT.png}} \\
	& \\
	TOS RWTH Aachen University & \noindent\parbox[c]{\hsize}{\includegraphics[scale=0.5]{../Logos/TOS.eps}} \\
	\vspace{1.5cm} & \\
	Würth Electronik GmbH \& Co. KG & \noindent\parbox[c]{\hsize}{\includegraphics[scale=0.05]{../Logos/Wuerth.png}}
\end{tabularx} \\

\newpage
\noindent
Die Produktionskosten unseres Gestikulasers können wie folgt aufgegliedert werden:

\subsubsection*{Oktokommander} 
\begin{tabularx}{\textwidth}{L{3.5cm} | S c S c}
	Produktname		 				& {Kosten \unit{€} / Stück}	& Anzahl & {Gesamtkosten \unit{€}}	& Kostenträger \\ 
	\hline
	Arduino Micro 					& 19.99		&   1    & 19.99 	& TOS 	 \\ [2mm]
	I2C-Multiplexer TCA9548A		& 7.80    	&   1    & 7.80  	& ILT    \\ [8mm]
	I2C ADS1015						& 11,20  	&   1    & 11.20 	& ILT    \\ [2mm]
	Infrarot LED    				& 5.60		&   4    & 22.40	& Würth  \\ [2mm]
	LED Treiber     				& 6.75    	&   2    & 13.50 	& Würth  \\ [2mm]
	Infrarot Photodiode SFH 203 FA	& 0.38		&   4    & 1.52  	& ILT	 \\ [15mm] 
	OP-Verstärker    				& 1.85    	&	4    & 7.40 	& TOS    \\ [2mm]
	USB 2 TypA Mount				& 0.89    	&   7    & 6.23  	& Würth  \\ [8mm]
	Passive Bauteile    			& {-------} & {---}  & 1.00	   	& TOS    \\ [2mm]
	\hline
									&			&		 & 91.04 	&
\end{tabularx}\\

\subsubsection*{Detektormodul} 
\begin{tabularx}{\textwidth}{L{3.5cm} | S c S c}
	Produktname 					& {Kosten \unit{€} / Stück}	& Anzahl & {Gesamtkosten \unit{€}}	& Kostenträger \\ 
	\hline
	I2C ADS1015						& 11.20  	&   1    & 11.20	& ILT    \\ [2mm]
	Infrarot Photodiode SFH 203 FA	& 0.38		&   4    & 1.52 	& ILT	 \\ [15mm]
	OP-Verstärker    				& 1.85   	&	4    & 7.40 	& TOS    \\ [2mm]
	USB 2 TypA Plug					& 0.50  	&   1    & 0.50 	& Würth  \\ [2mm]
	Passive Bauteile    			& {-------} & {---}	 & 1.00	   	& TOS    \\ [2mm]
	\hline
									&			&		 & 21.62	&		
\end{tabularx} \\

\noindent
Insgesamt wurden ein Oktokommander und sieben Detektormodule gebaut. Die Gesamtkosten für den Prototypen belaufen sich somit auf \unit[242.38]{€}. Darüber hinaus sind T-Shirts in Wert von \unit[200]{€} von Aconity3D gesponsert worden.