\chapter{Marketing und Kosten}
\label{ch:Kosten}

Neben der Entwicklung unseres Produktes wurde im Laufe der Entwicklungszeit zusätzliche Öffentlichkeitsarbeiten geleistet. So wurde die Website \url{https://www.gestikulaser.de/} ins Leben gerufen sowie der Instagram Account \href{https://www.instagram.com/laserharptos/}{\texttt{laserharptos}} mit über 100 Abonnenten betreut. 
Im Zuge dessen wurde auch Sponsoren angeworben, welche Interesse an der Unterstützung unseres Projektes hatten. \\

Folgende Sponsoren haben uns bei diesem Projekt unterstützt: \\

\begin{tabularx}{\textwidth}{r c}
	Aconity3D GmbH & \noindent\parbox[c]{\hsize}{\includegraphics[scale=0.5]{../figures/AconityLogo.png}} \\
	& \\
	Fraunhofer ILT & \noindent\parbox[c]{\hsize}{\includegraphics[scale=0.8]{../figures/ILTLogo.png}} \\
	& \\
	TOS RWTH Aachen University & \noindent\parbox[c]{\hsize}{\includegraphics[scale=0.5]{../figures/TOSLogo.png}} \\
	& \\
	Würth Electronik GmbH \& Co. KG & \noindent\parbox[c]{\hsize}{\includegraphics[scale=0.05]{../figures/WuerthLogo.png}}
\end{tabularx} \\

Die Produktionskosten unseres Gestikulasers können wie folgt aufgegliedert werden:\\

\large{\underline{\textbf{Oktokommander}}} \\ 

\normalsize
\begin{tabularx}{\textwidth}{p{4.5cm} | c c c c}

Produktname 			& Kosten /Stk 		& Anzahl & Gesamtkosten    & Kostenträger \\ \hline
Arduino Micro 			& $19.99$ \euro{}   &   1    & $19.99$ \euro{} & TOS 	\\ [2mm]
I2C-Multiplexer TCA9548A& $7.80$ \euro{}    &   1    & $7.80$ \euro{}  & ILT    \\ [8mm]
I2C ADS1015				& $11.20$ \euro{}   &   1    & $11.20$ \euro{} & ILT    \\ [2mm]
Infrarot LED    		&					&   4    &				   & Würth  \\ [2mm]
LED Treiber     		&                   &   2    &                 & Würth  \\ [2mm]
Infrarot Photodiode 	&					&   4    &                 & ILT	\\ [8mm] 
OP-Verstärker    		& $1.85$ \euro{}    &	4    & $7.40$ \euro{}  & TOS    \\ [2mm]
UBS 2 TypA -Mount		&                   &   7    &                 & Würth  \\ [2mm]
Passive Bauteile    	&	---------       &   ---  &	$1$ \euro{}	   & TOS    \\ [2mm]\hline
						&					&		 & $  60  $	\euro{}&

\end{tabularx}\\

\large{\underline{\textbf{Detektormodul}}} \\ 

\normalsize
\begin{tabularx}{\textwidth}{p{4.5cm} | c c c c}

Produktname 			& Kosten /Stk 		& Anzahl & Gesamtkosten    & Kostenträger \\ \hline
I2C ADS1015				& $11.20$ \euro{}   &   1    & $11.20$ \euro{} & ILT    \\ [2mm]
Infrarot Photodiode 	&					&   4    &                 & ILT	\\ [8mm]
OP-Verstärker    		& $1.85$ \euro{}    &	4    & $7.40$ \euro{}  & TOS    \\ [2mm]
UBS 2 TypA -Plug		&                   &   1    &                 & Würth  \\ [2mm]
UBS 2 TypA -Mount		&                   &   1    &                 & Würth  \\ [2mm]
Passive Bauteile    	&	-----  	        &	-	 &	$1$ \euro{}	   & TOS    \\ [2mm]\hline
						&					&		 & $  20  $	\euro{}&
					
\end{tabularx} \\

Dabei wurde ein Oktokommander und sieben Detektormodule aufgebaut. Die Gesamtkosten für die Produktion belaufen sich also auf 200€.\\

Darüber hinaus sind Kosten von BLA \euro{} für Unterkunft und Anreise entstanden sowie T-Shirts in Wert von BLA \euro{}. Die T-Shirts wurden von Aconity3D gespondert.