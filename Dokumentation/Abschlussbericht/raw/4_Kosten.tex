\chapter{Marketing und Kosten}
\label{ch:Kosten}

Neben der Entwicklung unseres Produktes wurde im Laufe der Entwicklungszeit zusätzliche Öffentlichkeitsarbeiten geleistet. So wurde die Website \url{https://www.gestikulaser.de/} ins Leben gerufen, sowie der Instagram Account \href{https://www.instagram.com/laserharptos/}{\texttt{laserharptos}} mit über 100 Abonnenten betreut. In Zusammenhang mit diesen Marketing Maßnahmen wurden auch Sponsoren angeworben, welche Interesse an der Unterstützung unseres Projektes hatten. \\
Folgende Sponsoren haben uns bei diesem Projekt unterstützt: \\

\begin{tabularx}{\textwidth}{r c}
	Aconity3D GmbH & \noindent\parbox[c]{\hsize}{\includegraphics[scale=0.07]{../Logos/AC3D_Logo_Print-for-white.png}} \\
	& \\
	Fraunhofer ILT & \noindent\parbox[c]{\hsize}{\includegraphics[scale=0.1]{../Logos/Fraunhofer_ILT.png}} \\
	& \\
	TOS RWTH Aachen University & \noindent\parbox[c]{\hsize}{\includegraphics[scale=0.5]{../Logos/TOS.eps}} \\
	\vspace{1.5cm} & \\
	Würth Electronik GmbH \& Co. KG & \noindent\parbox[c]{\hsize}{\includegraphics[scale=0.05]{../Logos/Wuerth.png}}
\end{tabularx} \\

\newpage
\noindent
Die Produktionskosten unseres Gestikulasers können wie folgt aufgegliedert werden:

\subsubsection*{Oktokommander} 
\begin{tabularx}{\textwidth}{p{4.5cm} | c c c c}
	Produktname		 			& Kosten / Stück 	& Anzahl & Gesamtkosten    & Kostenträger \\ 
	\hline
	Arduino Micro 				& \unit[19.99]{€}   &   1    & \unit[19.99]{€} & TOS 	\\ [2mm]
	I2C-Multiplexer TCA9548A	& \unit[7.80]{€}    &   1    & \unit[7.80]{€}  & ILT    \\ [8mm]
	I2C ADS1015					& \unit[11.20]{€}   &   1    & \unit[11.20]{€} & ILT    \\ [2mm]
	Infrarot LED    			&					&   4    &				   & Würth  \\ [2mm]
	LED Treiber     			&                   &   2    &                 & Würth  \\ [2mm]
	Infrarot Photodiode 		&					&   4    &                 & ILT	\\ [8mm] 
	OP-Verstärker    			& \unit[1.85]{€}    &	4    & \unit[1.85]{€}  & TOS    \\ [2mm]
	UBS 2 TypA -Mount			&                   &   7    &                 & Würth  \\ [2mm]
	Passive Bauteile    		&	---------       &  ---   & \unit[1]{€}	   & TOS    \\ [2mm]
	\hline
								&					&		 & \unit[60]{€}	   &
\end{tabularx}\\

\subsubsection*{Detektormodul} 
\begin{tabularx}{\textwidth}{p{4.5cm} | c c c c}
	Produktname 			& Kosten / Stück	& Anzahl & Gesamtkosten    	& Kostenträger \\ 
	\hline
	I2C ADS1015				& \unit[11.20]{€}   &   1    & \unit[11.20]{€}	& ILT    \\ [2mm]
	Infrarot Photodiode 	&					&   4    &                 	& ILT	 \\ [8mm]
	OP-Verstärker    		& \unit[1.85]{€}    &	4    & \unit[1.85]{€}  	& TOS    \\ [2mm]
	UBS 2 TypA -Plug		&                   &   1    &                 	& Würth  \\ [2mm]
	UBS 2 TypA -Mount		&                   &   1    &                 	& Würth  \\ [2mm]
	Passive Bauteile    	&	---------       &  ---	 & \unit[1]{€}	   	& TOS    \\ [2mm]
	\hline
							&					&		 & \unit[20]{€}		&		
\end{tabularx} \\

\noindent
Insgesamt wurden ein Oktokommander und sieben Detektormodule gebaut. Die Gesamtkosten für die Produktion belaufen sich also auf \unit[200]{€}. Darüber hinaus sind Kosten von BLA \euro{} für Unterkunft und Anreise entstanden sowie T-Shirts in Wert von BLA \euro{}. Die T-Shirts wurden von Aconity3D gesponsert.