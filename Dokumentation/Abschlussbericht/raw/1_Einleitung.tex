\chapter{Einleitung}
\label{ch:Einleitung}

Für Menschen sind Gesten schon seit frühester Zeit eine intuitive Art der Kommunikation. Bereits unsere Vorfahren haben sich auf der Jagd mittels verschiedener Gesten verständigt und bis heute nutzen wir Gesten in der zwischenmenschlichen Kommunikation, um uns z.B. mit Menschen zu verständigen, die eine andere Sprache sprechen. Da Gesten eine grundlegende Kommunikationsmöglichkeit bilden, wurde in den letzten Jahren viel geforscht, um diese Art der Kommunikation ebenfalls auf den Bereich der Mensch-Computer-Interaktion anwenden zu können. Mit der fortschreitenden Digitalisierung unseres Lebens bekommt dies einen immer wichtigeren Stellenwert in unserem Alltag.\\
Mit dem Gestikulaser haben wir ein neues Gestenerkennungssystem entwickelt, um statische Handgesten eines Menschen zu erkennen und diese zur Interaktion mit einem Endgerät zu nutzen. Dabei soll der Gestikulaser nicht mit einer Kamera arbeiten, wie die meisten heute verfügbaren Systeme, sondern stattdessen soll die Hand des Nutzers mit Infrarot-LEDs beleuchtet und die Gesten durch die erzeugten Reflektionsmuster erkannt werden. Eine auf diese Weise realisierte Gestenerkennung erlaubt eine deutlich kompaktere Bauweise und hat nur einen geringen Energieverbrauch, wodurch sie für den Einsatz in eingebetteten Systemen bestens geeignet ist. Zudem kann sie auch in vollkommener Dunkelheit betrieben werden, da lediglich Lichtreflektionen der IR-LEDs detektiert werden müssen. \\
Durch den Entwurf eines modularen Stecksystems aus mehreren Komponenten soll es möglich sein, die für die Detektion der Lichtreflektionen benötigten Photodioden künftig in einem individuellen Muster anordnen zu können, wodurch das Gestenerkennungssystem optimal auf den jeweiligen Anwendungsfall angepasst werden kann. Im Alltag kann der Gestikulaser in unterschiedlichen Bereichen eingesetzt werden. Für den Anfang kann er für einfache Aufgaben, wie z.B. die Steuerung eines ferngesteuerten Autos eingesetzt werden. Prinzipiell sind aber auch ein Einsatz zur Steuerung einer Smart Home Einrichtung oder sonstiger Endgeräte denkbar.