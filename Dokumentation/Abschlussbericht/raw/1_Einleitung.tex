\chapter{Einleitung}
\label{ch:Einleitung}

\textcolor{red}{Was wurde gemacht? Was war die Motivation? Inwiefern besitzt das Projekt eine Alltagsrelevanz?}

Gestenerkennung ist immer wieder ein Thema, welches viel Aufmerksamkeit erregt. Und obwohl der Mensch so einfach typische Gesten erkennen kann, bleibt es für den Computer eine große Herausforderung, zuverlässig die Gesten eines Menschen zuzuordnen. \\
Mit Gestikulaser wollen wir ein neues System entwickeln, um die Handgesten eines Menschen zu erkennen und diese individuell auf den Nutzer ab zu stimmen. Dabei soll es nicht, wie die meisten heute typischen Systeme, mit einer Kamera arbeiten, sondern durch Lichtreflexionen einer Hand die Geste zuordnen. Denn damit ist die Gestenerkennung nicht nur Tageslicht unabhängig, sondern kann auch in vollkommener Dunkelheit betrieben werden. \\
Mit Hilfe eines modularen Stecksystems mit mehreren Komponenten soll es möglich sein, ein individuelles Muster des Systems an zu fertigen. Somit soll in Zukunft nicht nur die Handgesten-Steuerung möglich sein, sondern auch eine ganzkörper Gestensteuerung wie sie in einem Cave verwendet wird ermöglicht werden. \\

Im Alltag soll der Gestikulaser dann eingesetzt werden, um verschiedenste Dinge, wie ein ferngesteuertes Auto, eine Smart Home Einrichtung oder eine Drohne zu steuern. 