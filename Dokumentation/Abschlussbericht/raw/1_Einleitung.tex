\chapter{Einleitung}
\label{ch:Einleitung}

\textcolor{red}{Was wurde gemacht? Was war die Motivation? Inwiefern besitzt das Projekt eine Alltagsrelevanz?}

Gestenerkennung ist immer wieder ein Thema, welches viel Aufmerksamkeit erregt. Und obwohl ein Mensch recht einfach verschiedene Gesten erkennen kann, ist es für den Computer eine große Herausforderung, zuverlässig die Gesten eines Menschen zu erkennen. \\
Mit dem Gestikulaser wollen wir ein neues System entwickeln, um Handgesten eines Menschen zu erkennen. Dabei soll dieses nicht mit einer Kamera arbeiten, wie die meisten heute verfügbaren Systeme, sondern die Hand des Nutzers soll mit Infrarot-LEDs beleuchtet werden und die Gesten sollen durch die erzeugten Reflektionsmuster erkannt werden. Eine auf diese Weise realisierte Gestenerkennung ist nicht nur Tageslicht unabhängig, sondern kann auch in vollkommener Dunkelheit betrieben werden. \\
Mit Hilfe eines modularen Stecksystems mit mehreren Komponenten soll es möglich sein, ein individuelles Muster des Systems an zu fertigen. Somit soll in Zukunft nicht nur die Handgesten-Steuerung möglich sein, sondern auch eine ganzkörper Gestensteuerung wie sie in einem Cave verwendet wird ermöglicht werden. \\

Im Alltag soll der Gestikulaser dann eingesetzt werden, um verschiedenste Dinge, wie ein ferngesteuertes Auto, eine Smart Home Einrichtung oder eine Drohne zu steuern. 