\chapter{Einleitung}
\label{ch:Einleitung}

\textcolor{red}{Was wurde gemacht? Was war die Motivation? Inwiefern besitzt das Projekt eine Alltagsrelevanz?}

Ziel unseres Projektes ist es, ein System zu entwickeln, welche Handgesten erkennt. Dabei sollen die Gesten individuell auf den Nutzer abgestimmt werden können, so dass eine möglichst genaue Gestenerkennung realisiert werden kann.

Bislang werden Gestenerkennungen meist mit Hilfe einer Kamera realisiert. Dabei wird ein Bild der Hand analysiert und dadurch eine Geste erkannt. Probleme: Bei Dunkelheit kann das System nicht eingesetzt werden. Gleiche Hintergrundfarbe wie die Hand ist nicht zu erkennen. 

Modulares Stecksystem mit mehreren Komponenten, die sich für jede Anwendung in einem individuellen Muster bzw. einer individuellen Größe zusammenstecken lassen.

Praxisrelevanz: 
1.) Berührungslose bedienung verschiedener Dinge wie ein ferngesteuertes Auto, Smart Home oder einer Drohne. 
2.) Nutzung im Cave zur ganzkörper Gestensteuerung

Bisher basieren die meisten Gestenerkennungen auf der Bilderkennung einer Kamera.