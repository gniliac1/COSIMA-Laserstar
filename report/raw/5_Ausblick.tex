\chapter{Ausblick}
\label{ch:Ausblick}

\textcolor{red}{Erweiterung um Sensorhandschuh zur verbesserten Erkennung der Daten. Überarbeitung des Designs um die Module besser bzw. anders zusammen zu stecken.}

An dieser Stelle möchten wir einen Ausblick geben, wie der von uns entwickelte Gestikulaser  weiterentwickelt werden könnte. 

Hier wäre zunächst einmal eine Verfeinerung der aktuell durchführbaren Gestenerkennung denkbar, die nicht nur die Stellung der Hand, sondern auch die Krümmung der einzelnen Finger berücksichtigt. Zu diesem Zweck wurde bereits ein Prototyp für einen Sensorhandschuh entwickelt. Dieser könnte während der Trainingsphase vom Nutzer dazu verwendet werden, feinere Gesten aufzunehmen, wobei neben den von den Detektormodulen detektierten Photoströmen zusätzlich die Daten der auf dem Sensorhandschuh befindlichen Module gespeichert werden. Aktuell kommen hier Dehnungssensoren für jeden Finger, sowie ein Gyroskop und ein Beschleunigungssensor zum Einsatz. Diese verfeinerten Gestendaten stellen nun natürlich eine neue Herausforderung an das Modell, das zur Auswertung der Photoströme im Live-Betrieb eingesetzt wird: Statt wie vorher den eingehenden Photoströmen nur das Klassenlabel einer Geste zuzuordnen, muss es nun in der Lage sein, auf Basis der gemessenen Photoströme die wahrscheinliche Lage der Hand, sowie die Krümmung der Finger vorauszusagen. Dieses Problem ist mathematisch deutlich schwieriger zu lösen, was nicht zuletzt an der deutlich höheren Anzahl an Parametern liegt, die bestimmt werden müssen, um eine ausreichende Aussagekraft zu gewährleisten. Aufgrund der höheren Anzahl an Parametern werden zudem auch mehr Daten benötigt, um das Modell zu trainieren.